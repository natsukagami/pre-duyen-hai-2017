\documentclass[11pt,a4paper,oneside]{article}
\usepackage[utf8]{vietnam}
\newcommand{\contestname}{Pre Duyên Hải 2017}
\usepackage{olymp}
\usepackage[dvips]{graphicx}
\usepackage{color}
\usepackage{colortbl}
%\usepackage{expdlist}
%\usepackage{mfpic}
%\usepackage{comment}

% Treat all pictures as MPS (metapost) when using PDFLaTeX tool
%\ifx\pdftexversion\undefined
%\else
%  \DeclareGraphicsRule{*}{mps}{*}{}
%\fi

%\textsc{\renewcommand{\contestname}{
%}

% Copied from 'amstex.tex' \bmod declaration
\def\bdiv{\mskip-\medmuskip\mkern5mu\mathbin{\mathrm{div}}\mkern5mu\mskip-\medmuskip}
% End of copied text

\renewcommand{\t}{\texttt}

\begin{document}


% Tell LaTeX to make the title.
%\maketitle
% Tell LaTex to double space your text.
%\doublespacing
%\section{Topic 5}
\begin{problem}{Chia hết!}{Standard Input}{Standard Output}{1 giây}{256}

Khánh là một con người yêu thích toán học. Hàng ngày, Khánh hay ngồi nghiên cứu những bài toán hóc búa
mà bao đời nay chưa có lời giải, ví dụ như phỏng đoán của Goldbach: mọi số chẵn lớn hơn 2 đều là tổng của
hai số nguyên tố; hay là tồn tại vô số cặp số nguyên tố sinh đôi hay không...

Mặc dù phát biểu các bài toán trên vô cùng đơn giản, song việc tìm lời giải cho chúng dường như là không 
thể đối với loài người bây giờ. Vì vậy, hàng ngày Khánh chỉ dành 5 phút để nghiên cứu những vấn đề hóc búa 
trên và dành thời gian nghiên cứu cái khác.

Ngoài ra, vẻ đẹp của toán học còn đến từ những bài toán đơn giản nữa. Ta có thể liệt kê một số bài toán như
sau: cho hai số $N$, $X$, tìm số lớn nhất không viết được dưới dạng $aN + bX$; cho hai số $N$, $X$, tìm số nhỏ nhất có
$N$ chữ số chia hết cho $X$; cho hai số $N$, $X$, tìm ước chung lớn nhất của $N$ và $X$; cho hai số $N$, $X$, tìm ước chung
lớn nhất của số fib thứ $N$ và số fib thứ $X$; ...

Để ý, vẻ đẹp của toán học không chỉ đến từ những thứ trừu tượng như tổng 3 góc của tam giác = 180 độ hay
trực tâm, trọng tâm, tâm đường tròn ngoại tiếp và tâm đường tròn 9 điểm thẳng hàng (đường thẳng Euler)...
mà còn hiện hữu trong cuộc sống xung quanh ta: tháp Eiffel có hình dạng rất đặc biệt: tầng một của tháp có
dạng tuyến tính, tầng 2 có dạng số mũ; hình dạng đặc biệt của tổ ong cho kết quả tiết kiệm nhất mà đạt được
dung tích lớn nhất; ... Do đó Khánh dành kha khá thời gian nghiên cứu các vấn đề này để tự sướng.

Hôm qua, Khánh đã dành thời gian để nghiên cứu vẻ đẹp trong cuộc sống rồi nên hôm nay Khánh sẽ làm việc đơn
giản, ví dụ như làm bài toán đơn giản 2. Bạn có thể làm nhanh hơn Khánh không?

\InputFile

\begin{itemize}
	\item Gồm 1 dòng duy nhất chứa 2 số $N$ và $X$.
\end{itemize}

\OutputFile

\begin{itemize}
	\item In ra 1 số duy nhất là số cần tìm. Nếu số đó không tồn tại in $-1$.
\end{itemize}

\Constraints
\begin{itemize}
	\item Subtask 1 (10\% số điểm):
	\begin{itemize}
		\item $1 \le N \le 7$
		\item $1 \le X \le 10^3$
	\end{itemize}
	\item Subtask 2 (10\% số điểm):
	\begin{itemize}
		\item $1 \le N \le 15$
		\item $1 \le X \le 10^7$
	\end{itemize}
	\item Subtask 3 (20\% số điểm):
	\begin{itemize}
		\item $1 \le N \le 15$
		\item $1 \le X \le 10^{12}$
	\end{itemize}
	\item Subtask 4 (20\% số điểm):
	\begin{itemize}
		\item $1 \le N \le 100$
		\item $1 \le X \le 10^{15}$
	\end{itemize}
	\item Subtask 5 (40\% số điểm):
	\begin{itemize}
		\item $1 \le N \le 10^5$
		\item $1 \le X \le 10^{15}$
	\end{itemize}
\end{itemize}

\Example

\begin{example}
\exmp{
3 214
}{
214
}%
\end{example}

\vspace{.5cm}

\end{problem}


\end{document}
